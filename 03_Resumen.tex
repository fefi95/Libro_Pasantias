\begin{center}
    \vspace{2cm}
    \textbf{
    DESARROLLO DEL MÓDULO PRINCIPAL Y ESTADÍSTICAS DE LA LIBRERÍA AUDITORÍAS TURPIAL
    }

    \vspace{2cm}

    Por:\\
    Stefani Carolina Castellanos Torres

    \vspace{2cm}

    \textbf{RESUMEN}\\

    \vspace{2cm}

\end{center}

Auditorías Turpial es una librería para Django que permite mantener un historial de las operaciones realizadas sobre alguna tabla de base de datos, a través de auditorías. La empresa Turpial Development, requiere de una librería que le permita acelerar el desarrollo de aplicaciones que necesiten un sistema de este tipo, y esté desarrollado bajo la arquitectura de microservicios para que sea fácil de integrar. Esta librería consta de tres módulos: Principal (\textit{Core}), Estadísticas y Reportes. Adicionalmente, estos microservicios puede ser actualizados mediante una herramienta de Integración Continua. En el presente informe se describen los procesos de diseño, implementación y pruebas de parte del módulo \textit{Core} y completamente el módulo Estadísticas de Auditorías Turpial, desarrollado como pasantía larga en Turpial Development. Así como, la implantación del sistema de Integración Continua. Como resultado, se lograron todos los objetivos planteados y otros originados durante la fase de construcción. \\

    \textbf{Palabras claves}: Auditoría de Sistemas, Librería, Django, Integración Continua, Microservicio.
