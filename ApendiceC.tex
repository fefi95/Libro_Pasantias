\chapter{Backlog del proyecto}

\section*{Núcleo}

\begin{longtable}{| c | p{8cm} | c |c | c |}
\hline
Código & Historia de usuario                                                                                                                                                                                & Prioridad & Riesgo & Story Points \\ \hline
C01    & Como programador, quiero indicar cuáles elementos de mi base de datos quiero auditar para poder indicar a la librería cuales modelos debe llevarle seguimiento.                                    & Alta      & Alta   & 5            \\ \hline
C02    & Como programador, quiero poder almacenar cuando un elemento auditable fue creado así podré saber cuándo, quién y qué fue ingresado en el sistema.                                                  & Alta      & Alta   & 5            \\ \hline
C03    & Como programador, quiero poder almacenar cuando un elemento auditable fue editado así podré saber cuándo, quién y cómo fue alterada la información en el sistema.                                  & Alta      & Alta   & 5            \\ \hline
C04    & Como programador, quiero poder almacenar cuando un elemento auditable fue eliminado así podré saber cuándo, quién y cuál la información fue borrada en el sistema.                                 & Alta      & Alta   & 5            \\ \hline
C05    & Como usuario, quiero disponer de listas de cada elemento auditable para poder observar el comportamiento de mi información.                                                                        & Alta      & Baja   & 2            \\ \hline
C06    & Como usuario, quiero que las listas de mis auditorías puedan ordenarse según autor, fecha de creación, acción efectuada sobre la data                                                              & Alta      & Baja   & 1            \\ \hline
C07    & Como usuario, quiero que las listas de mis auditorías puedan filtrarse según autor, fecha de creación, acción efectuada sobre la data; para así facilitar la búsqueda de auditorías en específico. & Alta      & Baja   & 1            \\ \hline
C08    & Como programador, quiero poder configurar el orden en que se desplegarán las columnas de mis listados, así mantener los patrones de experiencia de usuario de mi sistema.                          & Media     & Media  & 2            \\ \hline
C09    & Como programador, quiero contar con etiquetas personalizadas que me permitan incluir cualquier listado de las auditorías en mi sistema                                                             & Alta      & Media  & 3            \\ \hline
C10    & Como programador, quiero contar con un medio para personalizar el Look and Feel de mis listados de auditorías para ajustarlos al estilo de mi sistema.                                             & Media     & Alta   & 5            \\ \hline
C11    & Como usuario, quiero contar con un control de acceso a los listados de las auditorías y así solo brindar a un conjunto de usuarios las auditorías(Dividida en C11.1, C11.2 y C11.3)                & Media     & Baja   & 8            \\ \hline
C11.1  & Como usuario, quiero contar con un login para restringir el acceso a usuarios no autorizados a mis auditorías.                                                                                     & Media     & Baja   & 2            \\ \hline
C11.2  & Como usuario, quiero poder autorizar/desautorizar usuarios al acceso de auditorías.                                                                                                                & Media     & Baja   & 5            \\ \hline
C11.3  & Como usuario, quiero poder listar a todos los usuarios autorizados a acceder a las auditorías.                                                                                                     & Media     & Baja   & 1            \\ \hline
C12    & Como usuario, quiero que el sistema se actualice con la última versión de la librería por medio de integración continua para garantizar el soporte al sistema en donde se instale la librería.     & Alta      & Alta   & 5            \\ \hline
C13    & Como desarrollador quiero que se instalen automáticamente las dependencias necesarias en el sistema que se instale la librería.                                                                    & Alta      & Media  & 3            \\ \hline
\end{longtable}


\section*{Estadísticas}

\begin{longtable}{| c | p{8cm} | c |c | c |}
\hline
Código & Historia de usuario                                                                                                                 & Prioridad & Riesgo & Story Points \\ \hline
E01    & Como usuario quiero ver las estadísticas globales entre todas las acciones auditables a lo largo de un rango de tiempo.             & Alta      & Media  & 2            \\ \hline
E02    & Como usuario quiero disponer de gráficas con las estadísticas globales.                                                             & Media     & Media  & 2            \\ \hline
E03    & Como usuario quiero ver las estadísticas por acción auditable a lo largo de un rango de tiempo.                                     & Alta      & Media  & 2            \\ \hline
E04    & Como usuario quiero disponer de gráficas con las estadísticas por acción auditable.                                                 & Media     & Media  & 2            \\ \hline
E05    & Como usuario quiero ver las estadísticas por autor a lo largo de un rango de tiempo.                                                & Alta      & Media  & 2            \\ \hline
E06    & Como usuario quiero disponer de gráficas con las estadísticas por autor.                                                            & Media     & Media  & 1            \\ \hline
E07    & Como usuario quiero ver las estadísticas de una tabla de auditoría específica de la base de datos a lo largo de un rango de tiempo. & Alta      & Media  & 3            \\ \hline
E08    & Como usuario quiero disponer de gráficas con las estadísticas de una tabla de auditoría específica.                                 & Media     & Media  & 1            \\ \hline
\end{longtable}
