\chapter{\textbf{Marco teórico}}

\thispagestyle{empty}

\section{Auditoría}

Es una revisión sistemática para determinar si la calidad de las actividades cumplen con los acuerdos planificados y si estos están implementados efectivamente y son adecuados para alcanzar sus objetivos. Ofrecen una oportunidad de mejora al sistema y pueden ser llevadas a cabo para cumplir con normas regulatorias. Las auditorías se pueden aplicar a sistemas, procesos, programas o servicios y pueden ser internas (realizadas por la misma empresa) o externas (realizadas por proveedores). \cite{weinstein1997total}

\section{Acciones auditables}

Se considera una acción auditable todo flujo del sistema que cree, edite o borre alguna información sensible para el mundo de negocio. Son registradas incluyendo información sobre quién realizó la acción, qué acción se intentó realizar y cuándo ocurrió la acción.

\section{Microservicio}

Es un proceso autónomo, cohesivo e independiente que suele interactuar con otros componentes a través de mensajes \cite{Microservices1}. Se enfocan en resolver un único problema y funcionan de manera aislada; si se presentan una falla no se propaga y puede ser atendido más rápidamente. \cite{Microservices2} \\

Estos servicios se manejan de manera descentralizada y hacen uso de un despliegue completamente automatizado. Cada uno de ellos pueden ser escritos en diferentes lenguajes y tecnologías para almacenar información y aún así, interactuar y compartir información. \cite{Microservices3}

\section{Integración Contínua}

Es una práctica de desarrollo de software en la que los miembros del equipo combinan su trabajo de forma periódica, usualmente cada día. Cada integración está verificada por una herramienta automatizada que empaquete y pruebe el código para detectar errores lo más rápido posible \cite{Integracion_Continua}, de esta manera se puede mejorar la calidad del software y reducir el tiempo en validar y publicar actualizaciones.

\section{Pruebas automatizadas}

Las pruebas tienen como objetivo ejercitar el código para detectar errores y verificar que el software satisface los requerimientos especificados para asegurar su calidad. Estas son realizadas desde el punto de vista del usuario y las funcionalidades son probadas ingresando información y examinando la salida. \cite{Pruebas_Automatizadas}\\ 

Esta labor puede ser larga y repetitiva, por lo que es conveniente contar con herramientas que provean métodos que faciliten el proceso de escribir y  ejecutar casos de prueba y así, reducir significativamente el esfuerzo y tiempo invertido por los desarrolladores. \cite{Pruebas_Automatizadas} A esto se le conoce como automatización de pruebas. 

\section{Patrón Modelo-Vista-Controlador}

Este patrón de diseño asigna objetos en una aplicación a uno de tres roles: modelo, vista o controlador y define cómo se comunican entre ellos. La colección de objetos de cada tipo puede ser referido como una “capa”.\cite{MVC}

\begin{itemize}
    \item Modelo: controla el comportamiento y la data de la aplicación, responde las solicitudes de información (usualmente desde la vista) y las instrucciones de cambio de estado (usualmente desde el controlador).
    \item Vista: maneja la presentación de la información.
    \item Controlador: interpreta las entradas del usuario y le informa al modelo y/o vista para cambiar lo que sea apropiado.
\end{itemize}
\cite{MVC1}


Figura 1. Diagrama del patrón MVC\\

Es importante notar que la vista y el controlador dependen del modelo, sin embargo el modelo es independiente. Esta separación permite probar el modelo aparte de la presentación visual. 

\section{Patrón Modelo-Vista-Plantilla}

Es una modificación del patrón MVC en el cual la “vista” es la función de callback para pedir una URL en particular, porque esta define cual dato es presentado y su comportamiento, no como se ve. La vista delega a la “plantilla”  la presentación de la información.

En el caso de Django, el controlador es el \textit{framework} en sí: “la maquinaría que envía una petición a la vista apropiada, de acuerdo a la configuración de URL de Django” \cite{MVT}

\section{Señales}

Permiten a las aplicaciones ser notificadas cuando ocurra una acción en algún otro lugar, es decir, las señales permiten a ciertos emisores notificar a un conjunto de receptores que alguna acción está siendo ejecutada. \cite{Signals} En términos de software, son el análogo a interrupciones de hardware. \cite{Senales}

\section{Mixins}

Es un estilo de programación en el cual las unidades de funcionalidad son creadas en una clase y se incorporan en otras \cite{Mixins}. Pueden entenderse como “un subclase abstracta que puede ser usada para especializar el comportamiento de una variedad de padres”.  \cite{Mixins2} \\

Usualmente, los Mixins, definen nuevos métodos que realizan alguna acción y luego llaman a los métodos del padre correspondiente; pueden utilizarse en varias clases de la jerarquía y sin importar en qué clases sean usadas. \\

Hay varias razones para usar Mixins: extienden las clases existentes sin tener que editar, mantener o combinar el código fuente; mantienen el proyecto en componentes separados; facilitan la creación de nuevas clases con funcionalidades pre-fabricadas y que se puede combinar según sea la necesidad.\cite{Mixins}


% IMPORTANT NOTE: For the tables and the figures, The label should be inside the caption but %
%                 outside the text. The label should be meaningful to the figure or table.   %

% \section{Figures}

% Some examples of displaying the figures in {\LaTeX} \ldots\\
% One single figure \ldots e.g. Figure \ref{digsig}: % cross referencing for figure 1 %
% \begin{figure}[h!]
% \begin{center}
% \includegraphics[angle=0,width=3in]{daigramforthedigitalsignature.eps}
% \end{center}
% \vspace{-1cm}{\textbf{\caption{Block Diagram for the Digital
% Signature in Decimal Cryptosystem}\label{digsig}}}
% \addtocontents{lof}{\protect\addvspace{.5cm}}
% \end{figure}

% Two-in-One figure \ldots e.g. Figure \ref{2in1}: % cross referencing for figure 2 %
% \begin{figure}
% \begin{center}
% (a)\includegraphics[angle=0,width=2.5in]{klnexact1.eps}
% (b)\includegraphics[angle=0,width=2.5in]{klnVIM1.eps}
% \end{center}
% \vspace{-1cm}{\textbf{\caption{(a) The Exact Solution (b) The
% Numerical Results for $u(x,t)$ by means of 2-iterate VIM
% solution.}\label{2in1}}}
% \addtocontents{lof}{\protect\addvspace{.5cm}}
% \end{figure}

% \section{Tables}

% Some examples of creating tables in {\LaTeX} \ldots\\
% Simple table \ldots e.g. Table \ref{crypto}: % cross referencing for table 1 %
% \begin{table}[h!]
% \begin{center}
% {\textbf{\caption {Time for Encryption/Decryption of different
% length of Plaintexts}\label{crypto}}}
% \addtocontents{lot}{\protect\addvspace{.5cm}}
% \begin{footnotesize}
% \vspace{0.5cm}
% \begin{tabular}{|c|c|c|c|c|}                                                           \hline%
% \textbf{Decimal}     & \multicolumn{4}{|c|}{\textbf{Public key}}                     \\%
% \textbf{cryptosystem}& \multicolumn{4}{|c|}{\textbf{0.4921}}                         \\\hline%
%                      & \multicolumn{2}{|c|}{encryption}    & decryption       & time \\\hline%
% text                 & $c_1$       & $c_2$                 & text             & ms   \\\hline%
% 130519               & 0.6511      & 919.03                & 130519           & 15   \\%
% 133519               & 0.5511      & 998.04                & 133519           & 0    \\%
% 987654               & 0.7249      & 310.75                & 987654           & 15   \\\hline%
% \end{tabular}
% \end{footnotesize}
% \end{center}
% \end{table}

% Slightly more complex table \ldots e.g. Table \ref{BLFvLGFvFCvTP}: % cross referencing for table 2 %

% \begin{table}[h!]
% \begin{center}
% {\bf{\caption{Comparison of LGF with BLF, FC, and TP}\label{BLFvLGFvFCvTP}}}%
% \addtocontents{lot}{\protect\addvspace{.5cm}}
% \begin{scriptsize}
% \vspace{0.5cm}
% \begin{tabular}{|cc|c|c|c|c||cc|c|c|c|c|}\hline%
%      Class & \textit{n} & \textbf{BLF} & \textbf{LGF} & \textbf{FC} & \textbf{TP} &
%      Class & \textit{n} & \textbf{BLF} & \textbf{LGF} & \textbf{FC} & \textbf{TP} \\\hline%
%           &  20 & 1.09 & 1.03 & 1.06 & 1.05 &       &  20 & 1.00 & 1.00 & 1.00 & 1.00 \\%
%           &  40 & 1.12 & 1.04 & 1.08 & 1.06 &       &  40 & 1.40 & 1.40 & 1.40 & 1.40 \\%
%          I &  60 & 1.13 & 1.05 & 1.09 & 1.05 &    VI &  60 & 1.10 & 1.05 & 1.05 & 1.05 \\%
%           &  80 & 1.15 & 1.06 & 1.09 & 1.06 &       &  80 & 1.00 & 1.00 & 1.00 & 1.00 \\%
%           & 100 & 1.12 & 1.04 & 1.07 & 1.03 &       & 100 & 1.13 & 1.07 & 1.07 & 1.07 \\\cline{3-6} \cline{9-12}%
% \multicolumn{2}{|c|}{Average} & 1.122 & {\bf 1.044} & 1.078 & 1.050 &%
% \multicolumn{2}{|c|}{Average} & 1.127 & {\bf 1.103} & 1.104 & 1.104 \\\hline%
%           &  20 & 1.00 & 1.00 & 1.00 & 1.00 &       &  20 & 1.22 & 1.19 & 1.19 & 1.13 \\%
%           &  40 & 1.10 & 1.10 & 1.10 & 1.10 &       &  40 & 1.20 & 1.12 & 1.17 & 1.10 \\%
%         II &  60 & 1.10 & 1.05 & 1.05 & 1.00 &   VII &  60 & 1.20 & 1.10 & 1.18 & 1.12 \\%
%           &  80 & 1.07 & 1.07 & 1.03 & 1.07 &       &  80 & 1.20 & 1.10 & 1.17 & 1.11 \\%
%           & 100 & 1.06 & 1.03 & 1.03 & 1.00 &       & 100 & 1.19 & 1.09 & 1.17 & 1.11 \\\cline{3-6} \cline{9-12}%
% \multicolumn{2}{|c|}{Average} & 1.065 & 1.050 & 1.042 & {\bf 1.034} &%
% \multicolumn{2}{|c|}{Average} & 1.202 & 1.119 & 1.176 & {\bf 1.114} \\\hline%
%           &  20 & 1.20 & 1.06 & 1.18 & 1.06 &       &  20 & 1.23 & 1.15 & 1.16 & 1.16 \\%
%           &  40 & 1.22 & 1.13 & 1.16 & 1.11 &       &  40 & 1.22 & 1.16 & 1.19 & 1.16 \\%
%       III &  60 & 1.26 & 1.10 & 1.19 & 1.11 &  VIII &  60 & 1.19 & 1.09 & 1.18 & 1.11 \\%
%           &  80 & 1.27 & 1.10 & 1.15 & 1.10 &       &  80 & 1.19 & 1.10 & 1.16 & 1.11 \\%
%           & 100 & 1.23 & 1.08 & 1.13 & 1.08 &       & 100 & 1.19 & 1.09 & 1.17 & 1.12 \\\cline{3-6} \cline{9-12}%
% \multicolumn{2}{|c|}{Average} & 1.239 & 1.093 & 1.162 & {\bf 1.092} &%
% \multicolumn{2}{|c|}{Average} & 1.204 & {\bf 1.116} & 1.172 & 1.132 \\\hline%
%           &  20 & 1.00 & 1.00 & 1.00 & 1.00 &       &  20 & 1.01 & 1.01 & 1.00 & 1.01 \\%
%           &  40 & 1.00 & 1.00 & 1.00 & 1.00 &       &  40 & 1.02 & 1.02 & 1.01 & 1.02 \\%
%         IV &  60 & 1.10 & 1.15 & 1.10 & 1.10 &    IX &  60 & 1.01 & 1.01 & 1.01 & 1.01 \\%
%           &  80 & 1.10 & 1.10 & 1.10 & 1.07 &       &  80 & 1.01 & 1.01 & 1.01 & 1.01 \\%
%           & 100 & 1.13 & 1.07 & 1.07 & 1.03 &       & 100 & 1.01 & 1.01 & 1.01 & 1.01 \\\cline{3-6} \cline{9-12}%
% \multicolumn{2}{|c|}{Average} & 1.065 & 1.063 & 1.054 & {\bf 1.040} &%
% \multicolumn{2}{|c|}{Average} & 1.011 & 1.011 & {\bf 1.008} & 1.012 \\\hline%
%           &  20 & 1.15 & 1.09 & 1.08 & 1.06 &       &  20 & 1.15 & 1.20 & 1.15 & 1.20 \\%
%           &  40 & 1.18 & 1.10 & 1.10 & 1.11 &       &  40 & 1.13 & 1.07 & 1.09 & 1.08 \\%
%          V &  60 & 1.16 & 1.09 & 1.11 & 1.08 &     X &  60 & 1.14 & 1.08 & 1.09 & 1.09 \\%
%           &  80 & 1.17 & 1.09 & 1.11 & 1.08 &       &  80 & 1.14 & 1.06 & 1.06 & 1.06 \\%
%           & 100 & 1.16 & 1.08 & 1.10 & 1.08 &       & 100 & 1.11 & 1.07 & 1.07 & 1.06 \\\cline{3-6} \cline{9-12}%
% \multicolumn{2}{|c|}{Average} & 1.165 & 1.092 & 1.100 & {\bf 1.082} &%
% \multicolumn{2}{|c|}{Average} & 1.135 & 1.098 & {\bf 1.092} & 1.098 \\\hline\hline%
% & & & & & & \multicolumn{2}{|c|}{\bf AVERAGE} & 1.133 & 1.079 & 1.099 & {\bf 1.076} \\\hline%
% \end{tabular}
% \end{scriptsize}
% \end{center}
% \end{table}
