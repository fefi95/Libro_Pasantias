\chapter{\textbf{Marco metodológico}}

\thispagestyle{empty}

\section{Descripción de la metodología TAUP}

TAUP o Turpial Agile Unified Process es una metodología ágil creada por la empresa Turpial Development basada en los principios del "manifiesto ágil" que brinda a su equipo de trabajo una manera eficaz de llevar a cabo el desarrollo de \textit{software}. 

TAUP considera que la prioridad es la satisfacción del cliente mediante entregas continuas las cuales deben realizarse de manera frecuente, entre 2 semanas y 2 meses. Está conformada por tres fases de desarrollo: creación, construcción y transición.

\section{Fase de Concepción}

El objetivo de esta etapa es que el equipo de desarrollo profundice la
comprensión de los requisitos del sistema y validación de la arquitectura propuesta. 

\subsection{Historias de Usuario (HU)}

Es una técnica para expresar requerimientos que se caracteriza por ser sencilla de escribir, leer y evaluar. El usuario final o cliente le dará valor a esta HU ya que describe una funcionalidad que él necesita.\\

Las historias de usuarios se escriben en oraciones que poseen tres componentes indispensables: el rol del usuario que quiere realizar la acción, qué desea hacer y por qué. Ejemplo: *ejemplo*. Una buena HU está caracterizada por ser independiente, negociable, valiosa, estimable, pequeña y verificable (INVEST).\\

Adicionalmente, deben contar con criterios de aceptación bien definidos que brinden más detalle respecto a los aspectos de la Historia de Usuario que validan que se ha logrado completar. 

\section{Fase de Construcción}

\section{Fase de Transición}