\chapter{\textbf{Marco metodológico}}

\thispagestyle{empty}

\section{Descripción de la metodología TAUP}

TAUP es una metodología ágil creada por la empresa Turpial Development basada en los principios del ”manifiesto ágil” que brinda a su equipo de trabajo una manera eficaz de llevar a cabo el desarrollo de \textit{software}. En el apéndice A, se encuentra mayor detalle sobre el manifiesto ágil. \\

TAUP considera que la prioridad es la satisfacción del cliente mediante entregas continuas, las cuales deben realizarse en lapsos entre una semana y un mes. Está conformada por tres fases: concepción , construcción y transición. En el apéndice B se especifican los diferentes aspectos relacionados a la metodología.

\subsection{Fase de Concepción}

El objetivo de esta etapa es que el equipo de desarrollo profundice la comprensión de los requerimientos del sistema. Para ello, se escriben las Historias de Usuario (HU) con su respectiva prioridad y nivel de dificultad. Estas son establecidas por el cliente y el equipo, respectivamente, haciendo uso de la siguiente escala: alta, media alta, media, media baja y baja. También, el equipo, asigna \textit{Story Points} a cada HU que ponderan el esfuerzo que se requiere para culminarla.\\

Las HU cuentan con criterios de aceptación bien definidos que brindan más detalle respecto a los aspectos que validan su completitud y poseen pruebas de aceptación para asegurar que el producto cumpla con los estándares establecidos. Adicionalmente, deben cumplir con ciertos requisitos para que se comience a desarrollar (\textit{Definition of Ready} o DoR) y otros para que sea considerada como culminada (\textit{Definition of Done} o DoD).\\

Por último, se construye la lista de todas las Historias de Usuarios (\textit{Backlog}) en el orden que se van a ejecutar tomando en cuenta la prioridad y el riesgo de la tarea y se elabora un calendario en el cual se agregarán los \textit{Sprints} que sean necesarios para que todas las Historias de Usuarios sean cumplidas a cabalidad, llamado \textit{Release Plan} (Apéndice D).

\subsection{Fase de Construcción}

El enfoque de esta fase es desarrollar todo lo planteado en la Concepción, validando así, la arquitectura planteada. La metodología se basa en \textit{Sprints} o iteraciones que son bloques de tiempo de duración corta y fija, en los que al final se puede obtener un producto potencialmente entregable. \\

El primer día de cada \textit{Sprint} se lleva a cabo la Reunión de Planificación, en la que todos los miembros del equipo revisan lo que se tiene planteado en el \textit{Release Plan}, se verifica que cada una de esas Historias de Usuario estén en estado DoR para que sean divididas en tareas y se aclare cualquier duda acerca de lo que se va a realizar. \\

Durante el \textit{Sprint}, diariamente, se lleva a cabo una reunión corta llamada \textit{Daily Stand Ups Meeting}, en la cual se discute el estado de las tareas para el momento, que se va a hacer y cuáles obstáculos se han presentado en el desarrollo de alguna HU.\\

El último día de cada \textit{Sprint}, los miembros del equipo deben reunirse para realizar la Revisión de Iteración, en la cual estará presente el cliente para mostrarle todos los avances. Luego, se realiza la Reunión de Retrospectiva, en donde se discute que hizo bien, que se puede mejorar y a que se compromete.

\subsection{Fase de Transición}

En esta fase se probará todo lo desarrollado en la fase de Construcción. Estas pruebas funcionales y no funcionales, verifican que el proyecto pueda ser utilizado en un ambiente de producción, además se creará un manual para el usuario, el cual le dará la capacitación necesaria al cliente para poder utilizar sin problemas el sistema. También se validará la documentación, comprobando que todo el código esté correctamente explicado para que pueda ser entendido con facilidad en caso de que sea necesario realizar cambios.\\

Luego de que el sistema esté completo, se tendrá que desplegar en el ambiente de producción, al terminar este proceso se otorgará un tiempo prudencial al cliente para efectuar pruebas y se llevan a cabo las correcciones pertinentes.
