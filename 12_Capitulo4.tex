\chapter{\textbf{Marco tecnológico}}
\thispagestyle{empty}

\section{Python}
\section{Virtualenv}
\section{Django}
\section{HTML}
\section{Javascript}
\section{Pytest}
\section{Django-Graphos o Chart.js}
\section{PostgreSQL}
\section{MySQL}
\section{SQLite}
\section{JSON}
\section{Git}
\section{Jenkins}

% \section{Defining Theorem, Lemma, Corollary, and \ldots}\label{introchpter4}
% \framebox[\textwidth][c]{Command for \bf{Definition}:}
% \begin{verbatim}
% \begin{definition}{\bf:}
% ...
% \end{definition}
% \end{verbatim}
% \framebox[1.1\width][l]{\bf{For example:}}
% \begin{definition}{\bf:}
% Let \ldots and \ldots then \ldots
% \begin{equation}
% R(0.j_1 j_2  \cdots j_{k - 1} j_k  \cdots ) = \left\{
% \begin{array}{l}
%  0,~{\rm when} ~j_1 < 4,                                                  \\
%  1,~{\rm when} ~j_1 \ge 5 ~{\rm or~when}~ j_1 \ge 4 ~{\rm and} ~j_2 \ge 5 \\
% \end{array} \right.
% \end{equation}
% $ \forall~ 0.j_1 j_2  \cdots j_{k - 1} j_k  \cdots  \in (0,1)$
% {\rm where} $ j_i  \in Z,~i = 1,2, \cdots $.  {\rm We call} $R$
% {\rm the} {\bf rounding of\mbox{}f function}.
% \end{definition}

% \vspace{1cm}%
% \framebox[\textwidth][c]{Command for \bf{Proposition}:}
% \begin{verbatim}
% \begin{proposition}{\bf:}
% ...
% \end{proposition}
% \end{verbatim}
% \framebox[1.1\width][l]{\bf{For example:}}
% \begin{proposition}{\bf:}\\
% Given \ldots Thus \ldots
% \end{proposition}

% \framebox[\textwidth][c]{Command for \bf{Theorem}:}
% \begin{verbatim}
% \begin{theorem}{\bf:}
% ...
% \end{theorem}
% \end{verbatim}
% \framebox[1.1\width][l]{\bf{For example:}}
% \begin{theorem}{\bf:}{\rm \bf (name of the theorem if needed)}\\
% Let...
% \begin{equation} \label{eq2.1}
%  {\rm Round}\left[ {m \cdot (x/x)} \right] = m
% \end{equation}
% if and only if the number $\left[m \cdot (x/x)\right]$ takes the
% form of either
% \begin{compactenum}[i.]
%     \item \label{i} $m + 0.j_1 j_2  \cdots j_{r - 1} j_r  \cdots $, where $R(0.j_1 j_2  \cdots j_{r - 1} j_r  \cdots ) =
%     0$ or,
%     \item \label{ii} $(m - 1) + 0.h_1 h_2  \cdots h_r  \cdots$, where $R(0.h_1 h_2  \cdots h_r  \cdots ) =
%     1.$
% \end{compactenum}
% \end{theorem}

% \vspace{1cm}
% \framebox[\textwidth][c]{Command for \bf{Proof}:}
% \begin{verbatim}
% \begin{proof}
% ...
% \end{proof}
% \end{verbatim}
% \framebox[1.1\width][l]{\bf{For example:}}\\
% \begin{proof}
% We \ldots
% \begin{enumerate}
%     \item In condition ({\em\ref{i}}),...
%     \item In condition ({\em\ref{ii}}), ...
% \end{enumerate}
% In chapter \ref{literature}, section \ref{citation} we proved that
% \ldots
% \end{proof}

% \clearpage
% \framebox[\textwidth][c]{Command for \bf{Remark}:}
% \begin{verbatim}
% \begin{remark}{\bf:}
% ...
% \end{remark}
% \end{verbatim}
% \framebox[1.1\width][l]{\bf{For example:}}
% \begin{remark}{\bf:}
% A square is a rectangle.
% \end{remark}

% \vspace{1cm}
% \framebox[\textwidth][c]{Command for \bf{Note}:}
% \begin{verbatim}
% \begin{note}{\bf:}
% ...
% \end{note}
% \end{verbatim}
% \framebox[1.1\width][l]{\bf{For example:}}
% \begin{note}{\bf:}
% Value for $b$ is always equal to 1.
% \end{note}

% \vspace{1cm}
% \framebox[\textwidth][c]{Command for \bf{Example}:}
% \begin{verbatim}
% \begin{example}{\bf:}
% ...
% \end{example}
% \end{verbatim}
% \framebox[1.1\width][l]{\bf{For example:}}
% \begin{example}{\bf:}
% $2\times2$ is a square.
% \end{example}

% \clearpage
% \framebox[\textwidth][c]{Command for \bf{Lemma}:}
% \begin{verbatim}
% \begin{lemma}{\bf:}
% ...
% \end{lemma}
% \end{verbatim}
% \framebox[1.1\width][l]{\bf{For example:}}
% \begin{lemma}{\bf:}
% Suppose \ldots then \ldots
% \end{lemma}

% \vspace{1cm}
% \framebox[\textwidth][c]{Command for \bf{Corollary}:}
% \begin{verbatim}
% \begin{corollary}{\bf:}
% ...
% \end{corollary}
% \end{verbatim}
% \framebox[1.1\width][l]{\bf{For example:}}
% \begin{corollary}{\bf:}
% Assume \ldots hence \ldots
% \end{corollary}

% \vspace{1cm}%
% \framebox[\textwidth][c]{Command for \bf{Conjecture}:}
% \begin{verbatim}
% \begin{conjecture}{\bf:}
% ...
% \end{conjecture}
% \end{verbatim}
% \framebox[1.1\width][l]{\bf{For example:}}
% \begin{conjecture}{\bf:}
% If \ldots then \ldots else \ldots
% \end{conjecture}
