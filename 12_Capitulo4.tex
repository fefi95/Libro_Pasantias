\textit{mixins}\chapter{\textbf{Marco tecnológico}}

\thispagestyle{empty}

En este capítulo se describen los aspectos tecnológicos relevantes para la comprensión del proyecto, así como las herramientas utilizadas durante la pasantía.


\section{Python}

Es un lenguaje de programación interpretado y fácil de entender. Posee estructuras de datos de alto nivel y una aproximación sencilla al paradigma de programación orientado a objetos. Cuenta con una amplia variedad de librerías que fomentan la reutilización de código; también posee tipos de datos dinámicos y una sintaxis simple para facilitar el rápido desarrollo de aplicaciones
 \cite{Python_tutorial}

\section{Django}

Es un \textit{framework} de código abierto, escrito en Python y está basado en el patrón Modelo-Plantilla-Vista. Proporciona diversas funcionalidades reutilizables para desarrollar, rápidamente, aplicaciones Web escalables. \cite{MVT}\\ % agregar referencia si es necesario

Django está equipado con un conjunto de Clases y \textit{Mixins} que pueden ser heredados y proveen el comportamiento estándar de alguna acción particular(i.e crear un elemento). Adicionalmente, el programador puede crear otros nuevos para extender estas funcionalidades. \\

También incluye un despachador de señales que ayuda a las aplicaciones desacopladas a recibir notificaciones cuando alguna acción ocurra en algún otro lugar en el \textit{framework}. Algunos de los  eventos sobre los que notifica Django son la inserción, actualización y eliminación de elementos de la base de datos y migración de la misma.

\section{Pytest}

Es un \textit{framework} escrito en Python, que facilita la escritura de complejas pruebas de funcionalidad para aplicaciones y librerías y así, asegurar la calidad del \textit{software} que será entregado. Permite controlar la ejecución de pruebas automatizadas y comparar los resultados obtenidos con los esperados. \cite{pytest}

\section{JavaScript Object Notation (JSON)}

Es un formato para intercambiar datos, ligero, independiente del lenguaje y está basado en textos. Define un pequeño conjunto de reglas para la representación de datos estructurados que sean portables. JSON puede representar cuatro tipos primitivos (cadenas de caracteres, números, booleanos y null) y dos tipos estructurados (objetos y arreglos). \cite{JSON}

\section{PostgreSQL}

Es un poderoso sistema de base de datos relacional de código abierto. Su arquitectura goza de buena reputación gracias a su confiabilidad, integridad de los datos y correctitud. Puede ser implantada en la mayoría de los sistemas operativos y soporta claves foráneas, conjunciones, vistas, \textit{triggers}, procedimientos y la mayoría de los tipos de datos. \\

Este manejador de base de datos posee un tamaño ilimitado (depende del
hardware), con un máximo de 32 TB por tabla, 1 GB por campo, entre 250 y 1600
columnas por tabla y sin límites en la cantidad de índices por tablas
\cite{PostgreSQL}.

\section{MySQL}

Es sistema de base de datos relacional de código abierto que soporta múltiples plataformas y las todas la operaciones de SQL. Es muy rápido, confiable, escalable y fácil de usar. Soporta grandes volúmenes de datos, más de 50 millones de registros; puede manejar 200.000 tablas y hasta 64 índices por tablas. \cite{MySQL}

\section{SQLite}

Es una ligera librería \textit{in-process} que implementa un motor de base de datos SQL transaccional, de código abierto, que es autocontenido, no necesita configuración y no utiliza servidores puesto que escribe directamente a los archivos de disco. Posee todas las funcionalidades de una base de datos SQL completa: múltiples tablas, índices, \textit{triggers} y vistas.  Es multi-plataformas, se puede copiar libremente los archivos entre sistemas con diferentes arquitecturas. \\

Poseen un tamaño máximo de base de datos de 140 TB y por filas de 1 GB,  máximo 32767 columnas por tabla dependiendo del tipo de columna. La cantidad máxima de índices y tablas está estrechamente relacionada con la cantidad máxima de páginas (un poco más de 2 billones) ya que estos ocupan al menos una página del archivo de la base de datos. \cite{SQLite}

\section{Git}

Es un sistema de control de versiones de código abierto, diseñado para administrar cualquier tipo de proyecto con rapidez y eficiencia. Dispone de facilidades para llevar el seguimiento de los cambios realizados, soportar el desarrollo no-lineal, cambiar fácilmente de contexto y realizar experimentos sin afectar las versiones entregables del proyecto. \cite{Git}

\section{Jenkins}

Es un servidor de automatización de código abierto y de fácil instalación que puede ser utilizado para automatizar tareas relacionadas con el empaquetado, pruebas y despliegue de un \textit{software}. Puede ser utilizado como un servidor de Integración Continua en donde la versión más reciente del proyecto sea descargada, se ejecuten las tareas descritas anteriormente y si ocurre algún fallo se le notifique a los programadores. Adicionalmente, Jenkins dispone de un gran número de \textit{plugins}, por lo que se adapta a casi cualquier proyecto sin importar qué tecnología se estén usando. \cite{Jenkins}

\section{HyperText Markup Language (HTML)}

El lenguaje de marcado de hipertexto es un formato de datos simple, usado para crear documentos portables de una plataforma a otra y ha sido utilizada ampliamente en la World Wide Web desde 1990. \cite{RFC1866}

\section{Javascript}

Es un lenguaje interpretado, multi-paradigma y dinámico que soporta estilos de programación funcional,  orientado a objetos e imperativa, así como funciones de primera clase. Es comúnmente utilizado como el lenguaje de \textit{script} para páginas Web. \cite{javascript}

\section{jQuery}

Es una librería de JavaScript pequeña y rápida que simplifica la manipulación de documentos HTML, manejo de eventos, animaciones y AJAX. Cuenta con una API fácil de usar que funciona en una gran cantidad de navegadores.

\section{Amcharts}

Es una librería de JavaScript que permite añadir fácilmente gráficos
interactivos a los sitios Web y aplicaciones. Es compatible con todos los
navegadores modernos y la mayoría de los antiguos. Posee facilidades para crear
gráficos de Torta, Barras, Línea, Caja, \textit{Scatter} entre otros, y
cualquier combinación de ellos, exportar e importar los datos. También cuenta
con \textit{plugins} para extender su funcionalidad, es \textit{responsive},
fácil de personalizar y soporta múltiples lenguajes. \cite{Amcharts}

\section{Datatables}

Es un \textit{plugin} para la librería jQuery de JavaScript. Añade interacción
a cualquier tabla en HTML, dispone de funciones de búsqueda, paginación,
ordenamiento, filtros, entre otras. Provee facilidades para exportar a
diferentes tipos de archivo como CSV, PDF, XLS e incluso imprimir el contenido de la tabla.

\section{Bootstrap}
Es un \textit{framework} de código abierto para desarrollar rápidamente
aplicaciones, tanto Web como móviles, utilizando HTML, CSS y JavaScript.
También usa variables y \textit{mixins} de Sass, tiene un sistema de cuadrícula
(\textit{grid}) \textit{responsive}, gran cantidad de componentes pre-fabricados y poderosos plugins construidos con  jQuery.
