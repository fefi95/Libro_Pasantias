\chapter{\textbf{Marco tecnológico}}
\thispagestyle{empty}

\section{Python}

Es un lenguaje de programación interpretado y fácil de entender. Posee estructuras de datos de alto nivel y una aproximación sencilla al paradigma de programación orientado a objetos. Cuenta con una amplia variedad de librerías que fomentan la reutilización de código; también posee tipos de datos dinámicos y una sintaxis simple para facilitar el rápido desarrollo de aplicaciones.
 \cite{Python_tutorial}

\section{Django}

Es un \textit{framework} de código abierto, escrito en Python y está basado en el patrón Modelo-Plantilla-Vista. Proporciona diversas funcionalidades reutilizables para desarrollar, rápidamente, aplicaciones Web escalables. \cite{MVT}\\ % agregar referencia si es necesario

Existe comportamiento que es comúnmente utilizado en las aplicaciones (i.e crear un elemento), por esto, Django está equipado con un conjunto de Clases y Mixins que pueden ser heredados y proveen el comportamiento estándar de alguna acción. Adicionalmente, el programador puede crear otros nuevos para extender estas funcionalidades.\\

También incluye un despachador de señales que ayuda a las aplicaciones desacopladas a recibir notificaciones cuando alguna acción ocurra en algún otro lugar en el \textit{framework}. Algunos de los  eventos sobre los que notifica Django son la inserción, actualización y eliminación de elementos de la base de datos y migración de la misma.

\section{HyperText Markup Language (HTML)}

El lenguaje de marcado de hipertexto es un formato de datos simple, usado para crear documentos portables de una plataforma a otra y ha sido utilizada ampliamente en la World Wide Web desde 1990. \cite{RFC1866}

\section{Javascript}

Es un lenguaje interpretado, multi-paradigma y dinámico que soporta estilos de programación funcional,  orientado a objetos e imperativa, así como funciones de primera clase. Es comúnmente utilizado como el lenguaje de \textit{script} para páginas Web. \cite{javascript}

\section{Pytest}

Es un \textit{framework} escrito en Python, que facilita la escritura de complejas pruebas de funcionalidad para aplicaciones y librerías y así, asegurar la calidad del software que será entregado. Permite controlar la ejecución de pruebas automatizadas y comparar los resultados obtenidos con los esperados \cite{pytest}

\section{PostgreSQL}

Es un poderoso sistema de base de datos relacional de código abierto. Su arquitectura goza de buena reputación gracias a su confiabilidad, integridad de los datos y correctitud. Puede ser implantada en la mayoría de los sistemas operativos y soporta claves foráneas, conjunciones, vistas, triggers, procedimientos y la mayoría de los tipos de datos. 

\section{MySQL}
\section{SQLite}

\section{JavaScript Object Notation (JSON)}

Es un formato para intercambiar datos, ligero, independiente del lenguaje y está basado en textos. Define un pequeño conjunto de reglas para la representación de datos estructurados que sean portables. JSON puede representar cuatro tipos primitivos (cadenas de caracteres, números, booleanos y null) y dos tipos estructurados (objetos y arreglos). \cite{JSON}

\section{Git}

Es un sistema de control de versiones de código abierto, diseñado para administrar cualquier tipo de proyecto con rapidez y eficiencia. Dispone de facilidades para llevar el seguimiento de los cambios realizados, soportar el desarrollo no-lineal, cambiar fácilmente de contexto y realizar experimentos sin afectar las versiones entregables del proyecto. \cite{Git}

\section{Jenkins}

Es un servidor de automatización de código abierto y de fácil instalación que puede ser utilizado para automatizar tareas relacionadas con el empaquetado, pruebas y despliegue de un software. Puede ser utilizado como un servidor de Integración Contínua en donde la versión más reciente del proyecto sea descargada, se ejecuten las tareas descritas anteriormente y si ocurre algún fallo se le notifique a los programadores. Adicionalmente, Jenkins dispone de un gran número de \textit{plugins}, por lo que se adapta a casi cualquier proyecto sin importar qué tecnología se estén usando. \cite{Jenkins}

\section{Amcharts}

Es una librería de JavaScript que permite añadir fácilmente gráficos interactivos a los sitios Web y aplicaciones. Es compatible con todos los navegadores modernos y la mayoría de los antiguos. Posee facilidades para crear gráficos de Torta, Barras, Línea, Caja, Scatter entre otros, y cualquier combinación de ellos, exportar e importar los datos. También cuenta con plugins para extender su funcionalidad, es responsive, fácil de personalizar y soporta múltiples lenguajes. \cite{Amcharts}

% \section{Defining Theorem, Lemma, Corollary, and \ldots}\label{introchpter4}
% \framebox[\textwidth][c]{Command for \bf{Definition}:}
% \begin{verbatim}
% \begin{definition}{\bf:}
% ...
% \end{definition}
% \end{verbatim}
% \framebox[1.1\width][l]{\bf{For example:}}
% \begin{definition}{\bf:}
% Let \ldots and \ldots then \ldots
% \begin{equation}
% R(0.j_1 j_2  \cdots j_{k - 1} j_k  \cdots ) = \left\{
% \begin{array}{l}
%  0,~{\rm when} ~j_1 < 4,                                                  \\
%  1,~{\rm when} ~j_1 \ge 5 ~{\rm or~when}~ j_1 \ge 4 ~{\rm and} ~j_2 \ge 5 \\
% \end{array} \right.
% \end{equation}
% $ \forall~ 0.j_1 j_2  \cdots j_{k - 1} j_k  \cdots  \in (0,1)$
% {\rm where} $ j_i  \in Z,~i = 1,2, \cdots $.  {\rm We call} $R$
% {\rm the} {\bf rounding of\mbox{}f function}.
% \end{definition}

% \vspace{1cm}%
% \framebox[\textwidth][c]{Command for \bf{Proposition}:}
% \begin{verbatim}
% \begin{proposition}{\bf:}
% ...
% \end{proposition}
% \end{verbatim}
% \framebox[1.1\width][l]{\bf{For example:}}
% \begin{proposition}{\bf:}\\
% Given \ldots Thus \ldots
% \end{proposition}

% \framebox[\textwidth][c]{Command for \bf{Theorem}:}
% \begin{verbatim}
% \begin{theorem}{\bf:}
% ...
% \end{theorem}
% \end{verbatim}
% \framebox[1.1\width][l]{\bf{For example:}}
% \begin{theorem}{\bf:}{\rm \bf (name of the theorem if needed)}\\
% Let...
% \begin{equation} \label{eq2.1}
%  {\rm Round}\left[ {m \cdot (x/x)} \right] = m
% \end{equation}
% if and only if the number $\left[m \cdot (x/x)\right]$ takes the
% form of either
% \begin{compactenum}[i.]
%     \item \label{i} $m + 0.j_1 j_2  \cdots j_{r - 1} j_r  \cdots $, where $R(0.j_1 j_2  \cdots j_{r - 1} j_r  \cdots ) =
%     0$ or,
%     \item \label{ii} $(m - 1) + 0.h_1 h_2  \cdots h_r  \cdots$, where $R(0.h_1 h_2  \cdots h_r  \cdots ) =
%     1.$
% \end{compactenum}
% \end{theorem}

% \vspace{1cm}
% \framebox[\textwidth][c]{Command for \bf{Proof}:}
% \begin{verbatim}
% \begin{proof}
% ...
% \end{proof}
% \end{verbatim}
% \framebox[1.1\width][l]{\bf{For example:}}\\
% \begin{proof}
% We \ldots
% \begin{enumerate}
%     \item In condition ({\em\ref{i}}),...
%     \item In condition ({\em\ref{ii}}), ...
% \end{enumerate}
% In chapter \ref{literature}, section \ref{citation} we proved that
% \ldots
% \end{proof}

% \clearpage
% \framebox[\textwidth][c]{Command for \bf{Remark}:}
% \begin{verbatim}
% \begin{remark}{\bf:}
% ...
% \end{remark}
% \end{verbatim}
% \framebox[1.1\width][l]{\bf{For example:}}
% \begin{remark}{\bf:}
% A square is a rectangle.
% \end{remark}

% \vspace{1cm}
% \framebox[\textwidth][c]{Command for \bf{Note}:}
% \begin{verbatim}
% \begin{note}{\bf:}
% ...
% \end{note}
% \end{verbatim}
% \framebox[1.1\width][l]{\bf{For example:}}
% \begin{note}{\bf:}
% Value for $b$ is always equal to 1.
% \end{note}

% \vspace{1cm}
% \framebox[\textwidth][c]{Command for \bf{Example}:}
% \begin{verbatim}
% \begin{example}{\bf:}
% ...
% \end{example}
% \end{verbatim}
% \framebox[1.1\width][l]{\bf{For example:}}
% \begin{example}{\bf:}
% $2\times2$ is a square.
% \end{example}

% \clearpage
% \framebox[\textwidth][c]{Command for \bf{Lemma}:}
% \begin{verbatim}
% \begin{lemma}{\bf:}
% ...
% \end{lemma}
% \end{verbatim}
% \framebox[1.1\width][l]{\bf{For example:}}
% \begin{lemma}{\bf:}
% Suppose \ldots then \ldots
% \end{lemma}

% \vspace{1cm}
% \framebox[\textwidth][c]{Command for \bf{Corollary}:}
% \begin{verbatim}
% \begin{corollary}{\bf:}
% ...
% \end{corollary}
% \end{verbatim}
% \framebox[1.1\width][l]{\bf{For example:}}
% \begin{corollary}{\bf:}
% Assume \ldots hence \ldots
% \end{corollary}

% \vspace{1cm}%
% \framebox[\textwidth][c]{Command for \bf{Conjecture}:}
% \begin{verbatim}
% \begin{conjecture}{\bf:}
% ...
% \end{conjecture}
% \end{verbatim}
% \framebox[1.1\width][l]{\bf{For example:}}
% \begin{conjecture}{\bf:}
% If \ldots then \ldots else \ldots
% \end{conjecture}
