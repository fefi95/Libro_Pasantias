\chapter*{\textbf{INTRODUCCIÓN}}

Con el auge del desarrollo de \textit{software}, surgieron entes que controlan y supervisan el uso de la información y garantizan la confiabilidad de los mismos, por lo que algunas aplicaciones requieren mecanismos para registrar las operaciones ejecutadas. También se pueden utilizar estas trazas para caracterizar el comportamiento del programa a desarrollar.\\

La gran cantidad de clientes que necesitan un servicio de este tipo, ha llevado a la empresa Turpial Development a querer agilizar el desarrollo de un proyecto con estas características. Para esto, se han valido del uso de librerías de Django, sin embargo, ninguna de ellas cumple sus expectativas.\\

Motivado por esto, la empresa decidió construir una librería que lleve las trazas de auditorías del sistema en el cuál se instale. El objetivo de esta pasantía es construir parcialmente el módulo Principal (\textit{Core}) y completamente el módulo Estadísticas de la librería Auditorías Turpial.\\

Por otro lado, la empresa considera que esta librería debe utilizar una arquitectura de microservicios que facilite la inclusión de la misma, y que posea un servidor de Integración Continua. Este último con la intención de que la librería se encuentre disponible la mayoría del tiempo, sin afectar a los que la usen.\\

El presente informe está conformado por seis capítulos:
\begin{itemize}
    \item En el capítulo 1 se describe el entorno empresarial en donde se desarrolla el proyecto.
    \item En el capítulo 2 se presenta el problema y los objetivos de la pasantía.
    \item En el capítulos 3 se describen los aspectos teóricos relacionados con el proyecto que son indispensables para la compresión y desarrollo del mismo.
    \item En el capítulo 4 se detallan los aspectos tecnológicos que permiten que el proyecto sea llevado a cabo.
    \item En el capítulo 5 se muestra la metodología utilizada en la empresa, que fue creada por ellos mismos.
    \item En el capítulo 6 se explica el proceso de ejecución de la pasantía, así como la evolución del producto y las pruebas realizadas.
\end{itemize}

Luego, se ofrecen las conclusiones y recomendaciones que podría tomar en cuenta la empresa. Finalmente, se presentan todos los apéndices necesarios para profundizar sobre la información relevante del proyecto.
