\chapter{\textbf{Definición del problema}}

\thispagestyle{empty}

En este capítulo se explica la motivación de la empresa para desarrollar el
proyecto y la justificación del mismo. Además, se establecen los objetivos que
se esperan alcanzar durante la pasantía.

\section{Antecedentes}

En ocasiones, las aplicaciones necesitan un sistema que controle y supervise el uso de la información, bien sea para cumplir con regulaciones o para observar el comportamiento de la información considerada sensible para el negocio. Turpial Development desea contar con una librería confiable y de fácil instalación que provea estas funcionalidades, de manera que puedan ser ofrecidas como un servicio a sus clientes. Esta librería debe poder instalarse en el \textit{framework} escrito en Python, Django, puesto que la empresa estableció como estándar de desarrollo para sus aplicaciones, estas tecnologías. \\

Existen algunas librería disponibles que son capaces de realizar estos
procedimientos. Una de las más populares es django-reversions, una extensión de
Django que provee control de versiones para los modelos y la posibilidad de
restaurar la base de datos a una versión anterior. La empresa utilizó esta
librería en algunos proyectos, sin embargo, su integración delega demasiada
responsabilidad al programador por lo que es susceptible a errores humanos y
agrega una complejidad, que en muchos casos, es innecesaria para el
comportamiento que se desea lograr. \\

Además, de esta extensión, hay otras que no han sido utilizadas por Turpial
Development (Apéndice M) que ofrecen estas características, utilizando otras
funciones que provee el \textit{framework}, sin embargo los problemas explicados
anteriormente se mantienen.

\section{Planteamiento del problema}

Motivados por la cantidad de clientes que requieren un sistema de auditorías y
porque la extensiones de Django no son fáciles de implementar, Turpial
Development, se vio en la necesidad de crear su propia librería con la intención de agilizar
el desarrollo de proyectos con estas funcionalidades. Esta permitirá
llevar la trazabilidad de la información sensible del negocio a través de
auditorías de manera sencilla.\\

Adicionalmente, ninguna de las librerías disponibles cuentan con un módulo que
permita visualizar los datos obtenidos a través de estadísticas o reportes. El
desarrollador debe realizar esta labor por él mismo; analizando la información
recaudada y ajustando la estructura de datos para hacer uso de otras librerías
que dispongan de la capacidad de realizar gráficas o generar PDF. En este
sentido, se desea que el producto a desarrollar proporcione un módulo capaz de
realizar los cálculos necesarios para mostrar estadísticas referentes a la
información recaudada por las auditorías y sus respectivas gráficas; y un
módulo que genere reportes CVS o PDF de los diferentes listados y que sean
personalizables.

\section{Justificación}

La empresa Turpial Development decidió invertir en el desarrollo de una
librería para la gestión de auditorías que se ajuste tanto a las necesidades de
sus programadores, como a la de sus clientes; que sea fácil de integrar en otro proyecto,
permita identificar cuáles acciones del sistema deben ser
auditables y almacenar las auditorías en una base de datos relacional. Asimismo, debe
crear listados, generar reportes, mostrar estadísticas y facilitar la personalización su apariencia sin necesidad de que el
desarrollador cree las plantillas desde cero.

La empresa plantea que esta librería sea desarrollada bajo una arquitectura de
microservicios. De esta manera, podrá incluirse fácil y rápido dentro de las aplicaciones, a través de
configuraciones y ajustes menores que no interfieran con su correcto
funcionamiento. Se desarrollará como una aplicación en Django que se agregará
como un módulo más al sistema que lo instale.\\

Adicionalmente, se desea utilizar alguna herramienta automatizada para integrar
el trabajo de manera continua, empaquetar, probar y desplegar la librería de
manera que se actualice con rapidez y esté disponible en todo momento.
De igual manera, la empresa considera que es de vital importancia realizar pruebas
automatizadas que aseguren la calidad de sistema que se desea entregar y que
las pruebas finales se realicen sobre un proyecto de uso interno.

\section{Objetivo general}

Implementar, probar y presentar las funcionalidades de selección, gestión y
listados de auditorías y el módulo Estadísticas de la librería de Auditorías Turpial.

\section{Objetivos específicos}

\begin{itemize}
    \item Diseñar el módulo \textit{Core} de la librería Auditorías Turpial y construir parcialmente de dicho módulo, específicamente las funcionalidades de selección de modelos a auditar y gestión de las auditorías, bajo una arquitectura de microservicios.
    \item Diseñar y construir el módulo Estadísticas de la librería Auditorías Turpial, bajo una arquitectura de microservicios.
    \item Implantar un sistema de integración continua con el repositorio donde se dispondrá la librería Auditorías Turpial.
\end{itemize}
