\chapter*{GLOSARIO}

\textbf{AJAX}:  técnica de desarrollo web para crear aplicaciones interactivas. Estas aplicaciones se ejecutan en el cliente y mantiene la comunicación asíncrona con el servidor en segundo plano. De esta forma es posible realizar cambios sobre las páginas sin necesidad de recargarlas, mejorando la interactividad, velocidad y usabilidad en las aplicaciones.\\

\textbf{Backend}: parte de las aplicaciones que procesa la entrada desde la interfaz de usuario.\\

\textbf{Callback}:  cualquier código ejecutable que es suministrado como argumento a otro código. Este es ejecutado en cualquier momento.\\

\textbf{Candlestick}: tipo de gráfico financiero que es utilizado para describir la fluctuación de los precios. \\

\textbf{Fixture}: colección de datos que pueden ser utilizadas para poblar la base de datos.\\

\textbf{Framework}: conjunto estandarizado de conceptos, prácticas y criterios para enfocar un tipo de problemática particular que sirve como referencia, para enfrentar y resolver nuevos problemas de índole similar.\\

\textbf{Frontend}:  parte del software que interactúa con los usuarios.\\

\textbf{In-Process}: programa ejecutable que funciona como servicio para otro en lugar de la versión instalable. Ejemplo: DLL.\\

\textbf{Look-and-feel}: aspecto del sitio para el usuario y lo que siente cuando él al interactuar con el mismo.\\

\textbf{PIP}: repositorio de software (paquetes) para el lenguaje de programación Python.\\

\textbf{Plugin}: aplicación o programa que se relaciona con otro para agregarle una función nueva específica.\\

\textbf{Portable}: aplicación informática que puede ser utilizada, sin instalación previa, en un ordenador que posea el sistema operativo para el que fue programada.\\

\textbf{Query}:  consulta realizada contra una base de datos. Se usa para obtener datos, modificarlos o bien borrarlos.\\

\textbf{Queryset}: en Django, representa una colección de objetos de la base de datos. Puede tener uno o muchos filtros. En términos de SQL, es equivalente a una sentencia SELECT y los filtros equivalen a la cláusula WHERE.\\

\textbf{Responsive}: es una filosofía de diseño y desarrollo cuyo objetivo es adaptar la apariencia de las páginas web al dispositivo que se esté utilizando para visitarlas.\\

\textbf{Sass}: un metalenguaje de Hojas de Estilo en Cascada (CSS).\\

\textbf{Scatter}: diagrama matemático que utiliza coordenadas cartesianas para mostrar valores, típicamente de un conjunto de datos con dos variables.\\

\textbf{Sprints}:  bloques temporales cortos y fijos (iteraciones). Cada iteración tiene que proporcionar un resultado completo, un incremento de producto que sea potencialmente entregable.\\

\textbf{Triggers}: acciones (funciones) que se ejecutan cuando sucede algún evento sobre las tablas de una base de datos, a las que se encuentra asociado.

\clearpage
