\begin{table}[h]
\centering
\caption{Comparación de Jenkins vs Fabric.}
\label{tabla:6.1}
\begin{tabular}{| p{\textwidth/4} | p{\textwidth/3} | p{\textwidth/3} |}
\hline
                                                 & \textbf{Jenkins}                                                                                                                                   & \textbf{Fabric}                                                                                                                                              \\ \hline
\textbf{Descripción}                             & Un servidor de automatización que puede ser utilizado para automatizar cualquier clase de tarea cómo construir, probar y desplegar software        & Es una librería y una herramienta de línea de comandos para coordinar el uso de SSH para despliegues de aplicaciones o tareas de sistemas de administración. \\ \hline
\textbf{Código abierto}                          & Sí.                                                                                                                                                & Sí.                                                                                                                                                          \\ \hline
\textbf{Lenguaje}                                & Escrito en Java                                                                                                                                    & Escrito en Python                                                                                                                                            \\ \hline
\textbf{Extensible}                              & Si. Cuenta con una gran cantidad de \textit{plugins} para ampliar sus funcionalidades básicas y una tienda en donde pueden adquirirse                       & No.                                                                                                                                                          \\ \hline
\textbf{Interfaz gráfica}                        & Sí.                                                                                                                                                & No.                                                                                                                                                          \\ \hline
\textbf{Integración con Gitlab}                  & Sí.                                                                                                                                                & No                                                                                                                                                           \\ \hline
\textbf{Ejecutar un script}                      & Sí.                                                                                                                                                & Sí.                                                                                                                                                          \\ \hline
\textbf{Empaquetamiento y despliegue programado} & Si. Se puede configurar un horario para ejecutar algún script que contenga todas las instrucciones para empaquetar, probar y desplegar el sistema. & No. Se debe ejecutar el archivo de configuración a través de la consola.                                                                                     \\ \hline
\textbf{Observaciones}                           & Altamente recomendado por la comunidad.Puede generar reportes de las pruebas ejecutadas.                                                           & Fácil de aprender, de configurar y de instalar.                                                                                                              \\ \hline
\end{tabular}
\end{table}
