\begin{table}[h]
\centering
\caption{Modelo de datos de la tabla "AuditableAction"}
\label{tabla:1.3}
\begin{tabular}{| p{0.20\textwidth} | p{0.25\textwidth} | p{0.45\textwidth} |}
\hline
\textbf{Nombre del campo} & \textbf{Tipo de dato}                               & \textbf{Descripción}                                                                                                                                                                   \\ \hline
ID                        & \textit{AutoField} (Entero)                                        & Identificador del registro auditable. Entero de 32 bits. Los valores van desde -2147483648 a 2147483647.                      \\ \hline
author                    & \textit{ForeignKey}                                         & Referencia al usuario que realizó la acción.                                                                                                                                           \\ \hline
action                    & \textit{CharField}. Limitado a las opciones: CREATED,
UPDATED, DELETED. & El tipo de acción efectuada sobre un modelo auditable.                                                                                                                                 \\ \hline
datetime            & \textit{DateTimeField} & Fecha en la que se registró la acción.\\ \hline
model\_json\_old          & JSONField                                          & Antiguo valor. Un JSON que representa el valor que tenía el registro que se ve afectado por la operación. En caso de que la acción sea "creado", el JSON representará el objeto vacío. \\ \hline
model\_json\_new          & JSONField                                           & Nuevo valor. Un JSON que representa el valor que tiene el registro que se ve afectado por la operación. En caso de que la acción sea "eliminado", el JSON representará el objeto vacío \\ \hline
\end{tabular}

\end{table}
