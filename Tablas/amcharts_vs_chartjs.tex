\begin{table}[h]
\centering
\caption{Comparación de Amcharts vs Charts.js}
\label{tabla:6.2}
\begin{tabular}{| p{\textwidth/4} | p{\textwidth/3} | p{\textwidth/3} |}
\hline
                                              & \textbf{Amcharts}                                                       & \textbf{Chart.js}                                            \\ \hline
\textbf{Código abierto}                       & Sí.                                                                     & Sí.                                                          \\ \hline
\textbf{Tipos de gráficos}                    & Línea, área, barras, torta, \textit{Scatter}, Gantt, Radar, de vela, entre otras & Línea, área, barras, torta, \textit{Scatter}, Radar, entre otras.     \\ \hline
\textbf{Tecnología para mostrar los gráfico}  & SVG                                                                     & HTML5 Canvas                                                 \\ \hline
\textbf{Líneas de tendencia}                  & Sí.                                                                     & Sí.                                                          \\ \hline
\textbf{Responsive}                           & Sí.                                                                     & Sí.                                                          \\ \hline
\textbf{Exportar los datos}                   & Sí, mediante un \textit{plugin}.                                                 & No.                                                          \\ \hline
\textbf{Zoom}                                 & Sí.                                                                     & Sí. Utilizando un \textit{plugin}.                                    \\ \hline
\textbf{Soporta múltiples lenguajes}          & Sí                                                                      & No.                                                          \\ \hline
\textbf{Librería integrada con Django}        & No.                                                                     & Sí.                                                          \\ \hline
\textbf{Manejo de grandes volúmenes de datos} & Sí.                                                                     & No. Según opiniones de los usuarios, disminuye el desempeño. \\ \hline
\end{tabular}
\end{table}
